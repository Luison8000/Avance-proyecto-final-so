\documentclass[12pt]{article}

\usepackage[spanish]{babel}
\usepackage[utf8]{inputenc}
\usepackage[T1]{fontenc}
\usepackage{geometry}
\usepackage{graphicx}
\usepackage{hyperref}
\usepackage{listings}
\usepackage{xcolor}

\geometry{a4paper, margin=1in}

\title{Proyecto de Redes, Seguridad y Sistemas Distribuidos \\ en Red Hat Enterprise Linux}
\author{Luis Martínez del Campo}
\date{\today}

\begin{document}

\maketitle

\tableofcontents
\newpage

\section{Introducción}

El presente proyecto tiene como objetivo diseñar, configurar y asegurar una red de datos básica utilizando sistemas operativos Linux. Se implementaron servicios de red, mecanismos de seguridad y herramientas de sistemas distribuidos, con el fin de integrar conocimientos teóricos en un entorno práctico y funcional.

\section{Diseño de la Red}

Se diseñó una red compuesta por dos máquinas virtuales con Red Hat Enterprise Linux conectadas mediante una red interna.

\subsection{Topología}

\begin{itemize}
\item Servidor: 192.168.100.10
\item Cliente: 192.168.100.20
\item Máscara de red: 255.255.255.0
\end{itemize}

\section{Modelo OSI Aplicado}

\begin{itemize}
\item Capa Física: transmisión de datos mediante red virtual en VirtualBox
\item Capa de Enlace: direccionamiento MAC y tramas Ethernet
\item Capa de Red: direccionamiento IP estático
\item Capa de Transporte: uso de TCP para servicios
\item Capa de Sesión: establecimiento de sesiones SSH
\item Capa de Presentación: cifrado mediante TLS
\item Capa de Aplicación: servicios HTTP y SSH
\end{itemize}

\section{Configuración de Red}

Se asignaron direcciones IP estáticas a cada máquina editando los archivos de configuración de red y reiniciando el servicio NetworkManager.

La conectividad se verificó mediante los comandos \texttt{ping} y \texttt{traceroute}, confirmando la comunicación entre ambas máquinas.

\section{Servicios de Red}

Se instalaron y configuraron los siguientes servicios:

\begin{itemize}
\item SSH para acceso remoto seguro
\item HTTP para alojamiento de páginas web
\item HTTPS para comunicación cifrada
\end{itemize}

Se verificó el acceso remoto mediante SSH y la visualización de una página web desde el navegador.

\section{Seguridad}

Se implementaron medidas de seguridad fundamentales:

\begin{itemize}
\item Configuración de firewall con firewalld
\item Autenticación SSH mediante claves públicas y privadas
\item Deshabilitación de acceso por contraseña
\item Cifrado de comunicaciones mediante HTTPS
\end{itemize}

\section{Análisis de Vulnerabilidades}

Se utilizaron herramientas como:

\begin{itemize}
\item nmap para escaneo de puertos
\item lynis para auditoría del sistema
\end{itemize}

Se identificaron riesgos como puertos abiertos innecesarios y configuraciones por defecto, proponiendo como solución el endurecimiento del firewall y la desactivación de servicios no utilizados.

\section{Sistemas Distribuidos}

Se implementó NFS como sistema de archivos distribuido, permitiendo compartir directorios entre el servidor y el cliente.

Esto permitió el acceso transparente a archivos remotos, demostrando el concepto de transparencia en sistemas distribuidos.

\section{Modelo Peer-to-Peer}

Se utilizó BitTorrent para transferir archivos entre las máquinas, demostrando el modelo P2P donde cada nodo puede actuar como cliente y servidor simultáneamente.

\section{Análisis y Reflexión}

Se observaron ventajas como:

\begin{itemize}
\item Escalabilidad
\item Acceso remoto eficiente
\item Compartición de recursos
\end{itemize}

Y desventajas como:

\begin{itemize}
\item Complejidad de configuración
\item Riesgos de seguridad si no se aplican controles adecuados
\end{itemize}

\section{Conclusión}

El proyecto permitió integrar conocimientos de redes, seguridad y sistemas distribuidos en un entorno realista, demostrando la importancia de la configuración correcta, la protección de los servicios y la comunicación eficiente entre sistemas.

\end{document}
